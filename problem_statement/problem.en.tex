\problemname{Manhattan Ambulance}
When Albert Pierre Shannon (APS), a self-proclaimed utilitarian theorist, 
first came to New York, he was swooned by the rancid smell on the street 
and got carried over to a hospital by an ambulance. Lying on the stretcher in NYC traffic, 
he cannot help but dream the utilitarian dream that if the ambulances can knock over some 
pedestrians and stuff them right in, they could arrive at the hospitals faster and everyone 
will be taken care of. Being the smart person he is, APS starts to think about how practical 
his theory is. We know that each ambulance has a capacity k ($1 \leq k \leq 1000$),
which means that the ambulance can carry up to k patients (not including APS). 
Manhattan will be represented as $n*m$ grids ($n \leq 100, m \leq 100$).
Each grid has a number d ($1 \leq d \leq 10^9$), 
which is the distance an ambulance travels to it from any of the four neighbouring cells.
Each grid also has p ($0 < p \leq 1000$) pedestrians, which means that, were the 
ambulance choose to go to that cell, it would have to run over all p pedestrians 
and carry them all to the hospital. Roads are bidirectional and the ambulances can travel
to in all four directions.
Your job is to find the shortest path while making sure that the ambulances 
doesn’t have to carry more than k patients on the ambulance en route.

\section*{Input}
The first line has three integers $n$, $m$ and $k$ ($1 \leq n,m \leq 100, 1 \leq k \leq 1000$)
The next $n$ lines of $m$ integers describe the distance you have to travel 
in order to step onto the specific cell in the grid.
The following $n$ lines of $m$ integers describe the number of pedestrians on that cell.
The numbers in the first cell (i.e. $grid[0][0]$) of both grids will always be zero.

\section*{Output}
Output a single integer, the minimum distance you need to travel to get to the bottom-right cell
without exceeding the capacity. If that's impossible, output -1.